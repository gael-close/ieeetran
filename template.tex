\documentclass[letterpaper, 10 pt, journal, twoside]{IEEEtran}
\IEEEoverridecommandlockouts

% Quarto basics

\usepackage{color}
\usepackage{fancyvrb}
\newcommand{\VerbBar}{|}
\newcommand{\VERB}{\Verb[commandchars=\\\{\}]}
\DefineVerbatimEnvironment{Highlighting}{Verbatim}{commandchars=\\\{\}}
% Add ',fontsize=\small' for more characters per line
\usepackage{framed}
\definecolor{shadecolor}{RGB}{241,243,245}
\newenvironment{Shaded}{\begin{snugshade}}{\end{snugshade}}
\newcommand{\AlertTok}[1]{\textcolor[rgb]{0.68,0.00,0.00}{#1}}
\newcommand{\AnnotationTok}[1]{\textcolor[rgb]{0.37,0.37,0.37}{#1}}
\newcommand{\AttributeTok}[1]{\textcolor[rgb]{0.40,0.45,0.13}{#1}}
\newcommand{\BaseNTok}[1]{\textcolor[rgb]{0.68,0.00,0.00}{#1}}
\newcommand{\BuiltInTok}[1]{\textcolor[rgb]{0.00,0.23,0.31}{#1}}
\newcommand{\CharTok}[1]{\textcolor[rgb]{0.13,0.47,0.30}{#1}}
\newcommand{\CommentTok}[1]{\textcolor[rgb]{0.37,0.37,0.37}{#1}}
\newcommand{\CommentVarTok}[1]{\textcolor[rgb]{0.37,0.37,0.37}{\textit{#1}}}
\newcommand{\ConstantTok}[1]{\textcolor[rgb]{0.56,0.35,0.01}{#1}}
\newcommand{\ControlFlowTok}[1]{\textcolor[rgb]{0.00,0.23,0.31}{#1}}
\newcommand{\DataTypeTok}[1]{\textcolor[rgb]{0.68,0.00,0.00}{#1}}
\newcommand{\DecValTok}[1]{\textcolor[rgb]{0.68,0.00,0.00}{#1}}
\newcommand{\DocumentationTok}[1]{\textcolor[rgb]{0.37,0.37,0.37}{\textit{#1}}}
\newcommand{\ErrorTok}[1]{\textcolor[rgb]{0.68,0.00,0.00}{#1}}
\newcommand{\ExtensionTok}[1]{\textcolor[rgb]{0.00,0.23,0.31}{#1}}
\newcommand{\FloatTok}[1]{\textcolor[rgb]{0.68,0.00,0.00}{#1}}
\newcommand{\FunctionTok}[1]{\textcolor[rgb]{0.28,0.35,0.67}{#1}}
\newcommand{\ImportTok}[1]{\textcolor[rgb]{0.00,0.46,0.62}{#1}}
\newcommand{\InformationTok}[1]{\textcolor[rgb]{0.37,0.37,0.37}{#1}}
\newcommand{\KeywordTok}[1]{\textcolor[rgb]{0.00,0.23,0.31}{#1}}
\newcommand{\NormalTok}[1]{\textcolor[rgb]{0.00,0.23,0.31}{#1}}
\newcommand{\OperatorTok}[1]{\textcolor[rgb]{0.37,0.37,0.37}{#1}}
\newcommand{\OtherTok}[1]{\textcolor[rgb]{0.00,0.23,0.31}{#1}}
\newcommand{\PreprocessorTok}[1]{\textcolor[rgb]{0.68,0.00,0.00}{#1}}
\newcommand{\RegionMarkerTok}[1]{\textcolor[rgb]{0.00,0.23,0.31}{#1}}
\newcommand{\SpecialCharTok}[1]{\textcolor[rgb]{0.37,0.37,0.37}{#1}}
\newcommand{\SpecialStringTok}[1]{\textcolor[rgb]{0.13,0.47,0.30}{#1}}
\newcommand{\StringTok}[1]{\textcolor[rgb]{0.13,0.47,0.30}{#1}}
\newcommand{\VariableTok}[1]{\textcolor[rgb]{0.07,0.07,0.07}{#1}}
\newcommand{\VerbatimStringTok}[1]{\textcolor[rgb]{0.13,0.47,0.30}{#1}}
\newcommand{\WarningTok}[1]{\textcolor[rgb]{0.37,0.37,0.37}{\textit{#1}}}

\providecommand{\tightlist}{%
  \setlength{\itemsep}{0pt}\setlength{\parskip}{0pt}}\usepackage{longtable,booktabs,array}
\usepackage{calc} % for calculating minipage widths
% Correct order of tables after \paragraph or \subparagraph
\usepackage{etoolbox}
\makeatletter
\patchcmd\longtable{\par}{\if@noskipsec\mbox{}\fi\par}{}{}
\makeatother
% Allow footnotes in longtable head/foot
\IfFileExists{footnotehyper.sty}{\usepackage{footnotehyper}}{\usepackage{footnote}}
\makesavenoteenv{longtable}
\usepackage{graphicx}
\makeatletter
\def\maxwidth{\ifdim\Gin@nat@width>\linewidth\linewidth\else\Gin@nat@width\fi}
\def\maxheight{\ifdim\Gin@nat@height>\textheight\textheight\else\Gin@nat@height\fi}
\makeatother
% Scale images if necessary, so that they will not overflow the page
% margins by default, and it is still possible to overwrite the defaults
% using explicit options in \includegraphics[width, height, ...]{}
\setkeys{Gin}{width=\maxwidth,height=\maxheight,keepaspectratio}
% Set default figure placement to htbp
\makeatletter
\def\fps@figure{htbp}
\makeatother
\newlength{\cslhangindent}
\setlength{\cslhangindent}{1.5em}
\newlength{\csllabelwidth}
\setlength{\csllabelwidth}{3em}
\newlength{\cslentryspacingunit} % times entry-spacing
\setlength{\cslentryspacingunit}{\parskip}
\newenvironment{CSLReferences}[2] % #1 hanging-ident, #2 entry spacing
 {% don't indent paragraphs
  \setlength{\parindent}{0pt}
  % turn on hanging indent if param 1 is 1
  \ifodd #1
  \let\oldpar\par
  \def\par{\hangindent=\cslhangindent\oldpar}
  \fi
  % set entry spacing
  \setlength{\parskip}{#2\cslentryspacingunit}
 }%
 {}
\usepackage{calc}
\newcommand{\CSLBlock}[1]{#1\hfill\break}
\newcommand{\CSLLeftMargin}[1]{\parbox[t]{\csllabelwidth}{#1}}
\newcommand{\CSLRightInline}[1]{\parbox[t]{\linewidth - \csllabelwidth}{#1}\break}
\newcommand{\CSLIndent}[1]{\hspace{\cslhangindent}#1}

\makeatletter
\@ifpackageloaded{tcolorbox}{}{\usepackage[many]{tcolorbox}}
\@ifpackageloaded{fontawesome5}{}{\usepackage{fontawesome5}}
\definecolor{quarto-callout-color}{HTML}{909090}
\definecolor{quarto-callout-note-color}{HTML}{0758E5}
\definecolor{quarto-callout-important-color}{HTML}{CC1914}
\definecolor{quarto-callout-warning-color}{HTML}{EB9113}
\definecolor{quarto-callout-tip-color}{HTML}{00A047}
\definecolor{quarto-callout-caution-color}{HTML}{FC5300}
\definecolor{quarto-callout-color-frame}{HTML}{acacac}
\definecolor{quarto-callout-note-color-frame}{HTML}{4582ec}
\definecolor{quarto-callout-important-color-frame}{HTML}{d9534f}
\definecolor{quarto-callout-warning-color-frame}{HTML}{f0ad4e}
\definecolor{quarto-callout-tip-color-frame}{HTML}{02b875}
\definecolor{quarto-callout-caution-color-frame}{HTML}{fd7e14}
\makeatother
\makeatletter
\makeatother
\makeatletter
\makeatother
\makeatletter
\@ifpackageloaded{caption}{}{\usepackage{caption}}
\AtBeginDocument{%
\ifdefined\contentsname
  \renewcommand*\contentsname{Table of contents}
\else
  \newcommand\contentsname{Table of contents}
\fi
\ifdefined\listfigurename
  \renewcommand*\listfigurename{List of Figures}
\else
  \newcommand\listfigurename{List of Figures}
\fi
\ifdefined\listtablename
  \renewcommand*\listtablename{List of Tables}
\else
  \newcommand\listtablename{List of Tables}
\fi
\ifdefined\figurename
  \renewcommand*\figurename{Figure}
\else
  \newcommand\figurename{Figure}
\fi
\ifdefined\tablename
  \renewcommand*\tablename{Table}
\else
  \newcommand\tablename{Table}
\fi
}
\@ifpackageloaded{float}{}{\usepackage{float}}
\floatstyle{ruled}
\@ifundefined{c@chapter}{\newfloat{codelisting}{h}{lop}}{\newfloat{codelisting}{h}{lop}[chapter]}
\floatname{codelisting}{Listing}
\newcommand*\listoflistings{\listof{codelisting}{List of Listings}}
\makeatother
\makeatletter
\@ifpackageloaded{caption}{}{\usepackage{caption}}
\@ifpackageloaded{subcaption}{}{\usepackage{subcaption}}
\makeatother
\makeatletter
\@ifpackageloaded{tcolorbox}{}{\usepackage[many]{tcolorbox}}
\makeatother
\makeatletter
\@ifundefined{shadecolor}{\definecolor{shadecolor}{rgb}{.97, .97, .97}}
\makeatother
\makeatletter
\makeatother

% Common packages
\usepackage[T1]{fontenc}
\usepackage{lipsum}
\usepackage{amsmath}
\usepackage{amssymb}
\usepackage{physics}
\usepackage{siunitx}
\usepackage{graphicx}
\usepackage[hyphens]{url}
\usepackage{threeparttable}
\usepackage{xcolor}
\usepackage{float}
\usepackage{xcolor}
\usepackage{booktabs}
\usepackage{makecell}
\usepackage{orcidlink}
% monofont
\usepackage[scaled=0.8]{inconsolata}

%styled tabled generated from pandas
\usepackage{colortbl}
\usepackage{multirow}
\usepackage{stfloats} % https://tex.stackexchange.com/a/324358

% Fix table in 2-column format and enable wide table
% https://tex.stackexchange.com/a/224096
\makeatletter
% My tables
\newenvironment{mytable}[1][htbp]{
    \begin{figure}[#1]
    \onecolumn
    \begin{minipage}{0.5\textwidth}}
{
    \end{minipage}
    \twocolumn
    \end{figure}}

\newenvironment{mywidetable}[1][htbp]{
    \begin{figure*}[#1]
    \onecolumn
    \begin{minipage}{1.0\textwidth}}
{
    \end{minipage}
    \twocolumn
    \end{figure*}}
\makeatother



% Muted text (for filler text)
\newcommand\muted[1]{%
\bgroup
\hskip0pt\color{black!40!}%
#1%
\egroup
}

% color links
\usepackage{hyperref}
\hypersetup{ colorlinks, citecolor=teal, linkcolor=teal, urlcolor=teal}


\begin{document}

% author blocks
\author{
        First Author\(^1\)\orcidlink{0000-0000-0000-0000},     Second
Author\(^1\)\orcidlink{0000-0000-0000-0000},     Third Author\(^2\)
    \thanks{\(^1\)Stanford University. \(^2\)Some corporation.}
    \thanks{Released on: 1/1/22}
    \thanks{Extra footnote..}
}


% title
\title{A Sample Article for IEEE Transactions}
\maketitle

% abstract
\begin{abstract}
    This document is a sample illustrating the Quarto \texttt{ieeetran}
    template. It includes the key elements of a scientific articles
    (references, equations, figures, tables, code, cross references).
    The template enables the generation of IEEE-formatted article from a
    Jupypter notebook.
\end{abstract}

% body
\begin{tcolorbox}[enhanced jigsaw, opacityback=0, leftrule=.75mm, titlerule=0mm, breakable, coltitle=black, left=2mm, colback=white, colframe=quarto-callout-note-color-frame, toptitle=1mm, arc=.35mm, bottomrule=.15mm, toprule=.15mm, bottomtitle=1mm, title=\textcolor{quarto-callout-note-color}{\faInfo}\hspace{0.5em}{Note}, rightrule=.15mm, colbacktitle=quarto-callout-note-color!10!white, opacitybacktitle=0.6]

This is a template

\end{tcolorbox}

\hypertarget{sec-intro}{%
\section{Introduction}\label{sec-intro}}

\href{https://quarto.org/}{Quarto} is recent tool enabling the
generation of polished HTML page or PDF article from Jupyter notebook
\protect\hyperlink{ref-Close2022-dt}{{[}1{]}}. It can generate, with one
command and minimal configuration, PDF formatted according to journal
template via extensions---see the
\href{https://github.com/quarto-journals/}{list of journals supported}.
This sample article illustrates the formating achieved with the
\texttt{ieeetran} extension. This extension invokes the
\href{https://www.ctan.org/tex-archive/macros/latex/contrib/IEEEtran/}{official
release of Latex IEEEtran style}. Before submitting to a IEEE journal,
consult \protect\hyperlink{ref-Ieee2020-zv}{{[}2{]}} for editorial
guidelines and the journal or conference instructions.

\muted{\lipsum[1-2]}

\hypertarget{executable-code}{%
\section{Executable code}\label{executable-code}}

Quarto can not only format, but also execute on the fly Python code. It
enables \emph{``executable articles''}
\protect\hyperlink{ref-Lasser2020-wo}{{[}3{]}}. Figure~\ref{fig-polar}
illustrates this feature. The code, included in the article, generates
the figure on the fly when rendering the article. Note that the code can
be easily hidden, or folded (this only works in the HTML output though,
the PDF being static).

\begin{Shaded}
\begin{Highlighting}[]
\ImportTok{import}\NormalTok{ numpy }\ImportTok{as}\NormalTok{ np}
\ImportTok{import}\NormalTok{ matplotlib.pyplot }\ImportTok{as}\NormalTok{ plt}

\NormalTok{r }\OperatorTok{=}\NormalTok{ np.arange(}\DecValTok{0}\NormalTok{, }\DecValTok{2}\NormalTok{, }\FloatTok{0.01}\NormalTok{)}
\NormalTok{theta }\OperatorTok{=} \DecValTok{2} \OperatorTok{*}\NormalTok{ np.pi }\OperatorTok{*}\NormalTok{ r}
\NormalTok{fig, ax }\OperatorTok{=}\NormalTok{ plt.subplots(}
\NormalTok{  subplot\_kw }\OperatorTok{=}\NormalTok{ \{}\StringTok{\textquotesingle{}projection\textquotesingle{}}\NormalTok{: }\StringTok{\textquotesingle{}polar\textquotesingle{}}\NormalTok{\} }
\NormalTok{)}
\NormalTok{ax.plot(theta, r)}
\NormalTok{ax.set\_rticks([}\FloatTok{0.5}\NormalTok{, }\DecValTok{1}\NormalTok{, }\FloatTok{1.5}\NormalTok{, }\DecValTok{2}\NormalTok{])}
\NormalTok{ax.grid(}\VariableTok{True}\NormalTok{)}
\NormalTok{plt.show()}
\end{Highlighting}
\end{Shaded}

\begin{figure}[H]

{\centering \includegraphics{template_files/figure-pdf/fig-polar-output-1.png}

}

\caption{\label{fig-polar}A line plot on a polar axis.}

\end{figure}

\begin{figure*}

{\centering \includegraphics{fig-wide.png}

}

\caption{\label{fig-wide}A wide figure spanning the two columns}

\end{figure*}

\hypertarget{sec-feature}{%
\section{Other features}\label{sec-feature}}

Wide figure are supported, as demonstrated by Figure~\ref{fig-wide}, by
appending \texttt{fig-env="figure*"} to the figure statement.

\muted{\lipsum[1]}

\hypertarget{equation}{%
\subsection{Equation}\label{equation}}

Equation~\ref{eq-maxwell} shows a block of equations.

\begin{equation}\protect\hypertarget{eq-maxwell}{}{
 \begin{aligned}\nabla \times \vec{\mathbf{B}} -\, \frac1c\, \frac{\partial\vec{\mathbf{E}}}{\partial t} & = \frac{4\pi}{c}\vec{\mathbf{j}} \\   \nabla \cdot \vec{\mathbf{E}} & = 4 \pi \rho 
 \end{aligned} 
}\label{eq-maxwell}\end{equation}

\hypertarget{tables}{%
\subsection{Tables}\label{tables}}

\begin{mytable}[!t]

\hypertarget{tbl-parameters}{}
\begin{longtable}[]{@{}llllll@{}}
\caption{\label{tbl-parameters}Table of sensor
parameters}\tabularnewline
\toprule()
Parameter & Symbol & Min & Typ & Max & Unit \\
\midrule()
\endfirsthead
\toprule()
Parameter & Symbol & Min & Typ & Max & Unit \\
\midrule()
\endhead
Supply current & \(i_\mathrm{off}\) & - & - & 10 & mA \\
Hall sensitivity & \(S_\mathrm{H}\) & 0.2 & - & - & V/T \\
Effective nr. of bits & \(\mathrm{ENOB}\) & 12 & - & - & - \\
\bottomrule()
\end{longtable}

\end{mytable}

Tables are problematic due to 2-column nature of the IEEE class, but an
easy workaround exist. Table~\ref{tbl-parameters} illustrates a simple
table. Such table should be wrapped with a custom environment
\texttt{mytable} (or \texttt{mywidecolumn} for wide table spanning the 2
columns) via the
\href{https://github.com/quarto-ext/latex-environment}{extension
latex-environment}.

\begin{table*}[!b]
\centering
\caption{Selected stock correlation and simple statistics}
\label{tbl-styled}
\begin{tabular}{llrrrrrrrrl}
\toprule
 &  & \multicolumn{4}{|c|}{Equity} & \multicolumn{4}{|c|}{Stats} & Rating \\
 &  & \multicolumn{2}{|c|}{Energy} & \multicolumn{2}{|c|}{Consumer} & \multicolumn{4}{|c|}{} &  \\
 &  & \rotatebox{45}{BP} & \rotatebox{45}{Shell} & \rotatebox{45}{H\&M} & \rotatebox{45}{Unilever} & \rotatebox{45}{Std Dev} & \rotatebox{45}{Variance} & \rotatebox{45}{52w High} & \rotatebox{45}{52w Low} & \rotatebox{45}{} \\
\midrule
\multirow[c]{2}{*}{Energy} & BP & {\cellcolor[HTML]{FCFFA4}} \color[HTML]{000000} 1.00 & {\cellcolor[HTML]{FCA50A}} \color[HTML]{000000} 0.80 & {\cellcolor[HTML]{EB6628}} \color[HTML]{F1F1F1} 0.66 & {\cellcolor[HTML]{F68013}} \color[HTML]{F1F1F1} 0.72 & 32.2 & 1,034.8 & 335.1 & 240.9 & \color[HTML]{006400} \bfseries BUY \\
 & Shell & {\cellcolor[HTML]{FCA50A}} \color[HTML]{000000} 0.80 & {\cellcolor[HTML]{FCFFA4}} \color[HTML]{000000} 1.00 & {\cellcolor[HTML]{F1731D}} \color[HTML]{F1F1F1} 0.69 & {\cellcolor[HTML]{FCA108}} \color[HTML]{000000} 0.79 & 1.9 & 3.5 & 14.1 & 19.8 & \color{gray} \bfseries HOLD \\
\cline{1-11}
\multirow[c]{2}{*}{Consumer} & H\&M & {\cellcolor[HTML]{EB6628}} \color[HTML]{F1F1F1} 0.66 & {\cellcolor[HTML]{F1731D}} \color[HTML]{F1F1F1} 0.69 & {\cellcolor[HTML]{FCFFA4}} \color[HTML]{000000} 1.00 & {\cellcolor[HTML]{FAC42A}} \color[HTML]{000000} 0.86 & 7.0 & 49.0 & 210.9 & 140.6 & \color[HTML]{006400} \bfseries BUY \\
 & Unilever & {\cellcolor[HTML]{F68013}} \color[HTML]{F1F1F1} 0.72 & {\cellcolor[HTML]{FCA108}} \color[HTML]{000000} 0.79 & {\cellcolor[HTML]{FAC42A}} \color[HTML]{000000} 0.86 & {\cellcolor[HTML]{FCFFA4}} \color[HTML]{000000} 1.00 & 213.8 & 45,693.3 & 2,807.0 & 3,678.0 & \color{red} \bfseries SELL \\
\cline{1-11}
\bottomrule
\end{tabular}
\end{table*}

Data-rich tables can also be generated \textbf{programmatically} from
Python code, and rendered on the fly. As an example, we execute here the
example code from
\href{https://pandas.pydata.org/docs/reference/api/pandas.io.formats.style.Styler.to_latex.html\#pandas.io.formats.style.Styler.to_latex}{the
Pandas documentation} for a complicated styled table. The result is
rendered in Table~\ref{tbl-styled}. Clearly, generating such table in
plain Latex would be \textbf{painful}. And the Latex code would have to
be tweaked with each data update.

\hypertarget{conclusion}{%
\subsection{Conclusion}\label{conclusion}}

We have showed how the extension handles all key elements of scholarly
writing. With one line of code, a Jupyter notebook (or it text
equivalent) can be formatted into a IEEE article, almost ready for
submission. During the drafting, the HTML output should be preferred
(use option \texttt{-\/-to\ ieeetran-html} to render to HTML). It is
quicker, and figure/table placement is trivial. In addition, the HTML
format, with its dynamic features, opens up advanced features, like
interactive plots to engage with the readers
\protect\hyperlink{ref-Close2022-dt}{{[}1{]}}.

\hypertarget{bibliography}{%
\section*{References}\label{bibliography}}
\addcontentsline{toc}{section}{References}

\hypertarget{refs}{}
\begin{CSLReferences}{0}{0}
\leavevmode\vadjust pre{\hypertarget{ref-Close2022-dt}{}}%
\CSLLeftMargin{{[}1{]} }%
\CSLRightInline{G. Close, {``Technical writing and publishing
{Data-Rich} articles with quarto.''} Towards Data Science, Sep. 2022
{[}Online{]}. Available:
\url{https://towardsdatascience.com/technical-writing-and-publishing-data-rich-articles-with-quarto-d61a56bcaa64}}

\leavevmode\vadjust pre{\hypertarget{ref-Ieee2020-zv}{}}%
\CSLLeftMargin{{[}2{]} }%
\CSLRightInline{IEEE, {``{IEEE} editorial style manual for authors.''}
Apr. 2020 {[}Online{]}. Available:
\url{https://journals.ieeeauthorcenter.ieee.org/your-role-in-article-production/ieee-editorial-style-manual/}}

\leavevmode\vadjust pre{\hypertarget{ref-Lasser2020-wo}{}}%
\CSLLeftMargin{{[}3{]} }%
\CSLRightInline{J. Lasser, {``Creating an executable paper is a journey
through open science,''} \emph{Communications Physics}, vol. 3, no. 1,
pp. 1--5, Aug. 2020 {[}Online{]}. Available:
\url{https://www.nature.com/articles/s42005-020-00403-4}}

\end{CSLReferences}

% done
\end{document}
